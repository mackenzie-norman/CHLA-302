\documentclass[professionalfont]{beamer}
\usepackage{newtxtext,newtxmath}
%Information to be included in the title page:
%\begin{frame}
%\frametitle{}
%\end{frame}
%\begin{frame}[fragile]
  %\begin{lstlisting}
       %%CODE HERE
   %\end{lstlisting}
%\end{frame}

\usepackage{listings}
\usepackage{xcolor}

\definecolor{codegreen}{rgb}{0,0.6,0}
\definecolor{codegray}{rgb}{0.5,0.5,0.5}
\definecolor{codepurple}{rgb}{0.58,0,0.82}
\definecolor{backcolour}{rgb}{0.95,0.95,0.92}

\lstdefinestyle{mystyle}{
    backgroundcolor=\color{backcolour},   
    commentstyle=\color{codegreen},
    keywordstyle=\color{magenta},
    numberstyle=\tiny\color{codegray},
    stringstyle=\color{codepurple},
    basicstyle=\ttfamily\footnotesize,
    breakatwhitespace=false,         
    breaklines=true,                 
    captionpos=b,                    
    keepspaces=true,                 
    numbers=left,                    
    numbersep=5pt,                  
    showspaces=false,                
    showstringspaces=false,
    showtabs=false,                  
    tabsize=2
}

\lstset{style=mystyle}
\title{Academic Article Presentation}
\subtitle{Latinidad by Marta Caminero-Santangelo}
\author{Mackenzie Norman}



\begin{document}

\frame{\titlepage}
\begin{frame}
\frametitle{Synopsis}
\textit{Latinidad} by Marta Caminero-Santangelo is a summarization and in-turn an interrogation of opinions about the various terms used to attempt to describe the Latino/a literary traditions and communities. 

Marta begins by simply describing different viewpoints on various terms (Hispanic, Chicano/a, Latino/a), and introducing the idea of panethnic groups.
\end{frame}
\begin{frame}
\frametitle{Synopsis}
Quickly though, she lays bare much of the problem at the core of these panethnic labels - especially when some of them hinge on a shared identity of colonialism.
\end{frame}
\begin{frame}
\frametitle{Key Points}

\end{frame}
\begin{frame}
``While the idea that “Latino” contrasts with “Anglo-American” is regarded as fairly self-evident, what is less immediately obvious is that “Latino” also potentially contrasts with terms like “Chicana,” “Cuban-American,” or “Puerto Rican.” According to this understanding, what “latinidad” suggests is a Latin-American-heritage identity that crosses boundary lines among the various specific national-origin groups, and implies a panethnic group. It implies, in other words, that a “Latino” identity exists that binds these various groups together in some tangible, substantive ways – ways that matter. It is much like using a term such as “Indian” instead of “Cherokee,” “Chickasaw,” or “Choctaw”; it suggests a grouping together of groups that, historically, have not always seen themselves as belonging to that larger group identity. For example, in 2006, only 23 percent of Latinos said that all Latinos share a single common culture, while an astounding 75 percent said that they did not (Suro and Escobar 2006: 10)''
\end{frame}

\begin{frame}
\frametitle{Discussion Questions}
\end{frame}

\begin{frame}
\frametitle{Discussion Questions}
As primary texts shed light on the arbitrariness of borders and boundary, how does it affect national pride \& identity? 
\end{frame}

\end{document}