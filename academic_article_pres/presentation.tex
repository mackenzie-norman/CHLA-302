\documentclass[professionalfont]{beamer}
\usepackage{newtxtext,newtxmath}
%Information to be included in the title page:
%\begin{frame}
%\frametitle{}
%\end{frame}
%\begin{frame}[fragile]
  %\begin{lstlisting}
       %%CODE HERE
   %\end{lstlisting}
%\end{frame}

\usepackage{listings}
\usepackage{xcolor}

\definecolor{codegreen}{rgb}{0,0.6,0}
\definecolor{codegray}{rgb}{0.5,0.5,0.5}
\definecolor{codepurple}{rgb}{0.58,0,0.82}
\definecolor{backcolour}{rgb}{0.95,0.95,0.92}

\lstdefinestyle{mystyle}{
    backgroundcolor=\color{backcolour},   
    commentstyle=\color{codegreen},
    keywordstyle=\color{magenta},
    numberstyle=\tiny\color{codegray},
    stringstyle=\color{codepurple},
    basicstyle=\ttfamily\footnotesize,
    breakatwhitespace=false,         
    breaklines=true,                 
    captionpos=b,                    
    keepspaces=true,                 
    numbers=left,                    
    numbersep=5pt,                  
    showspaces=false,                
    showstringspaces=false,
    showtabs=false,                  
    tabsize=2
}

\lstset{style=mystyle}
\title{Academic Article Presentation}
\subtitle{Latinidad by Marta Caminero-Santangelo}
\author{Mackenzie Norman}



\begin{document}

\frame{\titlepage}

\begin{frame}
\frametitle{Synopsis}
\textit{Latinidad} by Marta Caminero-Santangelo is a summarization and in-turn an interrogation of opinions about the various terms used to attempt to describe the Latino/a literary traditions and communities. 

Marta begins by simply describing different viewpoints on various terms (Hispanic, Chicano/a, Latino/a), and introducing the idea of panethnic groups.
\end{frame}
\begin{frame}
\frametitle{Synopsis}
Quickly though, she lays bare much of the problem at the core of these panethnic labels - especially when some of them hinge on a shared identity of colonialism.

However, Caminero-Santangelo also notes that there is a burgenoning group ``expanding their imaginative reach to grapple with the experiences of other Latino/a groups, mirroring the emergence of experiential latinidad''

In short, its complicated.
\end{frame}
\begin{frame}
\frametitle{Key Points}
\begin{itemize}
    \item ``what `latinidad' suggests is a Latin-American-heritage identity that crosses boundary lines among the various specific national-origin groups, and implies a panethnic group. \dots it suggests a grouping together of groups that, historically, have not always seen themselves as belonging to that larger group identity.''
    
    \item ``Even critics adamantly opposed to essentialized or reductive ideas of latinidad may argue for more nuanced understandings of panethnic Latino identity. Flores, for instance, proposes that “it is possible to find a common thread in the intricate ‘Latino’ weave, or at least a framework in which to interpret the … Latino presence in some more encompassing way”; one possible “framework” entails the “larger international context, Latin America''
\end{itemize}
\end{frame}
\begin{frame}
    \frametitle{Key Points}
\begin{itemize}
   \item ``in 2006, only 23 percent of Latinos said that all Latinos share a single common culture, while an astounding 75 percent said that they did not (Suro and Escobar 2006: 10)''
   \item ``Their collective bond might well be based on a common Latina identity that binds them together''
   
\end{itemize}
\end{frame}

\begin{frame}
\frametitle{Connection to primary text}
\textit{Latinidad} connects wonderfully to \textit{The Book of Unknown Americans} by Christina Henríquez. Henríquez herself is panamanian, but many of the characters in her ``Book of Unknown Americans'' come from places other than Panama. In her work we can see what Caminero-Santangelo labels ``imaginative reach'' wherein Henríquez does (masterfully) ``grapple with the experiences of other Latino/a groups''
\end{frame}

\begin{frame}
\frametitle{Discussion Questions}
\begin{itemize}
    \item As primary texts shed light on the arbitrariness of borders and boundary, how does it affect identity? Especially when identity is based on lines drawn by colonizers
    \item How does Henríquez in her novel use the described ``imaginative reach''. Does she fall short? What about other authors? Especially those such as American Dirt author Jeanine Cummins?
\end{itemize}
\end{frame}


\end{document}