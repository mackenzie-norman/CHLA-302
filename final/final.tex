\documentclass{article}
\usepackage{ifpdf} 
\usepackage[style =mla, backend=biber]{biblatex}
\addbibresource{research.bib}
\usepackage{mla}
\title{Midterm Essay}
\author{Mackenzie Norman}
\begin{document}
\begin{mla}{Mackenzie}{Norman}{Herrera}{CHLA302}{\today}{Questions of Agency in \textbf{The Book Of Unknown Americans}}

\noindent\textit{``Alas", said the mouse, ``the whole world is growing smaller every day. At the beginning it was so big that I was afraid, I kept running and running, and I was glad when I saw walls far away to the right and left, but these long walls have narrowed so quickly that I am in the last chamber already, and there in the corner stands the trap that I am running into." \linebreak \linebreak ``You only need to change your direction," said the cat, and ate it up.''} (\cite{eine-fable})

\section*{Introduction}
Freedom is a central tenet of the so-called American Dream. However, agency - defined as the ability for your choices to  meaningfully change the world around you -- is what I like to think of as the `why' behind this freedom. If your choices mean nothing then are you free, and if you are not free then are you even american? These difficult questions are what Christina Henríquez asks in \textbf{The Book Of Unknown Americans}. Existing in the genre defined as life-writing since ``in one way or another a person's life unfolds within the narrative''(\cite{testimonio}, 311 ). \textbf{The Book Of Unknown Americans} closely follows two immigrant families -- the Rivieras and the Toros -- living in an Apartment complex in Maryland. Narrated primarily by Alma Riveria, the matriarch of the Riviera family and Mayor Toro, the Toro's adolescent son. The story begins with the arrival of the Rivieras in an unknown apartment complex, where it is quickly exposited that they speak no English between the three of them and we begin to learn about Maribel's disability. The reader is then introduced to the Toros, first through Mayor and then Rafael briefly. Quickly the Toros and the Riverias become friends, with Mayor developing a crush on the daughter Maribel, this is not without reservation, since there is ``something wrong with her'' (\cite{Henriquez2014-sh},43) . As the novel progresses, Mayor and Maribel romance blossoms, with him even defending her from the looming villain Garret. However, things quickly deteriorate, as america plunges into recession and both patriarchs lose their jobs, which causes the non-citizen Riviera's to lose their work visa, rendering them illegal immigrants. During this time Mayor is grounded for fighting Garret and more importantly, lying about his role on the soccer team. This is extended after Quisqueya (the perennially nosey neighbor) tells his mother that she saw him   leaving the car with Maribel with semen on his pants. The novel then rachets up its pace considerably, Mayor steals his parents new car to take Maribel to see the coast while its snowing. This causes alarm with both the Riveria's and the Toros. In this panic, Alma confesses to Arturo that she saw Garret attempting to assault Maribel. Arturo, going alone, rides the bus over to search for Maribel where the story climaxes in his senseless killing at the hands of Garret's alcoholic father. The story focuses thematically on guilt, the american dream and love. However Henríquez also spends considerable time questioning ideas of agency, especially ``the double bind of the immigrant’s presence in the United States without rights to inclusion in the social or political sphere'' (\cite{Lutes_Travis_2021}). This essay will explore how Henríquez destabilizes the idea agency through use of idiosyncratic and often contradictory equivalencies of guilt, vehicle ownership, freudian conflict, and consent.


``The myth of American travel and mobility has long shaped ideas of nation and national identity. In postwar cultural production, to move freely is to enact rights governed by U.S. citizenship'' (\cite{ruiz2015transit},2). The car ``became equated early in the American cultural imagination with ... personal reinvention and self-determination'' (\cite{Uhlman2015-qx}, 1) A symbol of freedom ``Those who could control their own movement were deemed self-sufficient, independent agents'' (\cite{Uhlman2015-qx},2) Much of the initial conflict in the story is based on the dependence of public transit - Maribel's initial assault at the hands of Garret, is because A. Alma gets on the wrong bus and gets lost trying to get home in time to pick Maribel up from the bus, and B. because of the unyielding timetable of the school bus that drops her off. So it represents a change when the (citizen) Toros, after receiving money from an aunts alimony, purchase a Volkswagen. After purchasing it we immediately see the first instance of the double bind. Despite the new symbolic freedom the Toro's appear have acquired, Rafael drives far slower than the other cars going ``twenty-five in a fifty'' (\cite{Henriquez2014-sh}, 164), describing that despite their new vehicle; they are still stuck halfway. Continuing this Rafael explains


\noindent\linebreak
``You don't understand,'' my dad said. ``They stop you''
\noindent\linebreak
``Who? What are you talking about?'' my mom asked.
\noindent\linebreak
``Thats why I was being cautious.''
\noindent\linebreak
``Who stops you?''
\noindent\linebreak
``The police. If you're white, or maybe Oriental, they let you drive however you want. But if you're not, they stop you.''
\paragraph{}
Here, through Rafael's plaintive speech Henríquez lays bare the inherent contradiction in car ownership, that is so symbolic of the double-bind of agency in america. Even though he is promised ``The unrestrained capacity to move'' (\cite{Uhlman2015-qx}, 1) he must do so at half the pace of his white counterparts or face police violence. As such his capacity to move is restrained (despite being a citizen and now having a car) and so Henríquez raises a question of if he even fits what we have defined as an ``independent agent''(\cite{Uhlman2015-qx}, 1).



The primary contradiction of agency in the novel is Mayors conflict with his father. Mayor says he 
\textit{``wasn’t allowed to claim the thing I felt and I didn’t feel the thing I was supposed to claim''}(\cite{Henriquez2014-sh}, 109). This is one of the first acknowledgements of freudian psychology in the text, which at its heart is a psychology of ``the inner contradiction in liberty'' (\cite{loves-bdy}, 6 ).
Mayor, ``crucified by the contradictory commands issuing from the Freudian super-ego, which says both "thou shalt be like the father," and "thou shalt not be like the father" (\cite{loves-bdy}, 6 ), Henríquez affirms this through the car - a symbol of manhood; ``Do you have anything Italian?'' (161) asks Rafael. Affirming the ``equation of  true manhood and ... foreign luxury vehicles''(\cite{Uhlman2015-qx}, 19). After Mayor brings Maribel into the car (against his fathers wishes), he quickly adopts it as his own manhood, likening her actions in the car ``she grazed her fingertips across the center console, her nails scraping the hard plastic, I ... felt the blood pounding'' (\cite{Henriquez2014-sh}, 173) to Maribel grazing his own `manhood'; ``and then my pants were damp and warm'' (\cite{Henriquez2014-sh}, 175). This equivalency starts Mayor and his father on a collision course, ``Sonship and brotherhood are espoused against fatherhood'' (\cite{loves-bdy}, 6). This conflict escalates as Rafael symbolically castrates Mayor as punishment for using his manhood, ``You're not going to see her again''(\cite{Henriquez2014-sh}) since Maribel is ``The only girl who had ever liked me''(\cite{Henriquez2014-sh},264) this is effectively castration in the eyes of Mayor, prevented from seeing the only person with whom he could reproduce.  Creating  ``simultaneously a social reality and a legal impossibility – a subject \dots without rights'' (\cite{Lutes_Travis_2021}). This castration is symbolic, but also furthers the idea of being free, yet unable to enact change - a castrated man cannot reproduce and as such cannot become a patriarch - a violent removal from the ecstasy of agency he felt before. ``a look of envy. Dreams of possession'' (\cite{wretched}, 4). Mayor yearns, to be free, but the Henríquez frames it so that we the reader know that to be free is to be whole (un-castrated) and as such wholly able to affect the world and ``give her the thing that it seemed like everyone else wanted to keep from her'' (\cite{Henriquez2014-sh}, 231) 

Henríquez furthers the idea of envy with Mayors `stealing' his fathers car to take Maribel on a date to see the Ocean. We know from the beginning Mayor is a character acutely aware of the ``double bind''; ``I wasn’t allowed to claim the thing I felt and I didn’t feel the thing I was supposed to claim''(\cite{Henriquez2014-sh}, 109)  So when he takes his fathers car, the reader is acutely aware of the apparent impending danger. However in Mayors mind, he can only think of one thing highlighted here.

\textit{``Maribel and I deserved to be together and she deserved to see the snow if she wanted to and nobody was going to hold us back. I was her one chance. I wanted to give her the thing that it seemed like everyone else wanted to keep from her: freedom '' (\cite{Henriquez2014-sh}, 231)}

Stealing the car sets into motion the events that lead to the hate killing of Arturo, described by Mayor as either ``completely random, just something that happened'' (\cite{Henriquez2014-sh}, 261), or pre-determined ``on a path'' (\cite{Henriquez2014-sh}, 261) I do not see this as shirking responsibility, but rather the realization that, even in his pursuit (and success) in achieving freedom, there still lies a contradiction. In america, you can never have the freedom to change things unless ``you're white, or maybe Oriental.''(\cite{Henriquez2014-sh}) the only change comes from ``a language of pure violence''(\cite{wretched}, 4). Mayor knows this and even is appreciative of his own violence; ``Had I really punched him? But instead of feeling pain or any kind of remorse, I felt exhilarated'' (\cite{Henriquez2014-sh}, 128) This, and Arturos murder shows how the only true agent is the one with the power to enact violence. ``The Rule of Law is the Rule of Force.'' (\cite{loves-bdy}, 18) None of the immigrants wield this power despite being free (and in some cases in the right) to do so and as such. Henríquez places them as free but without agency. ``a group with a stake in national politics, but no voice.'' (\cite{Lutes_Travis_2021})

Henríquez uses guilt and fault as the primary motivation of the story, many of the characters struggle with fault, but it is most exposed in Maribels startling lucidity at the end of the novel.

\noindent\linebreak
``Do you think it was my fault what happened?'' 
\noindent\linebreak
...
\noindent\linebreak
I looked at her face. I could see she was going to live with that question for a long time. I'd been living with it for less than a day myself and it was already tearing me up. But I said the only thing I could. ``No. It was just what happened. That's all'' (\cite{Henriquez2014-sh},260)
\paragraph{}
Mayor says ``it just happened'', and we as the reader feel emotionally this to be true. But its not or not all the way true -- it is the contradiction, the so-called double bind again, If Mayor is free to do ``whatever I wanted''(\cite{Henriquez2014-sh}, ) then his acts of freedom must have affect, but by his (and the other main characters) admission, his actions can change nothing, because ``It was just what happened.''(\cite{Henriquez2014-sh},260)

In the novel we never get Maribel's perspective - this is intentional. Maribel is totemic, suffering from a closed head injury,  ``mute and immobile, a tree rooted in place'' (\cite{Henriquez2014-sh}, 122) she is unable to affect the world around her at all. Far from being human she is characterized as a plantlike object by Mayor that ``grew on me'' (\cite{Henriquez2014-sh}, 108). Her and Mayors relationship is the perverse realization of this double-bind of liberty. Mayor views himself as her savior: ``I was the only one who understood her, the only one who was willing to give her a chance.'' (\cite{Henriquez2014-sh}, 109), in doing he so he takes the role of america, adopting the mythos of american exceptionalism ``the best [special education] schools were in the United States''(\cite{Henriquez2014-sh}, 106), the equivalency is continued in the oft quoted line ``I wanted to give her the thing that it seemed like everyone else wanted to keep from her: freedom'' (\cite{Henriquez2014-sh},231 ). Mayor is taking the role america took quite literally in central america. The irony insoforth of the double-bind is that her and Mayors romance is framed by the author as consensual and supporting (even beautiful) -- but the support which Mayor provides is quite conditional on her romance, which we see when Maribel says she's going home, Mayor responds with ``So thats it.''(\cite{Henriquez2014-sh}, 262) vocalizing what america means by ``freedom'', you are only free so long as you are useful to me ``The only reason they sponsored our visas was because the government was pressuring them''(\cite{Henriquez2014-sh}, 181) and when value can no longer be extracted, you are harshly cast aside ``You really want to know? I was fired.''(\cite{Henriquez2014-sh}, 180) The perverse contradiction of Maribels consent to her relationship with Mayor is the background information that Maribel is mentally handicapped, we know she has an IEP and cannot even remember basic date and times. As such there must be serious questions of consent. Mayor views himself as giving her freedom, but he never has asked if freedom is something she wants, and the bigger question is whether she can even say (in an ethical sense) if she wants to be free, in fact we know she eventually learns to reject Mayors objectification ``Finding is for things that are lost. You don't need to find me, Mayor''(\cite{Henriquez2014-sh}, 263). By use -- and implied rejection of the term `thing', rejecting it and replace thing with me in the rejection, Maribel finally and definitively says no. This also causes her to juxtapose with Mayor, who deeply, violently yearns to be ``where no one could ground me, and where I could do whatever I wanted''(\cite{Henriquez2014-sh}, ), because she does not appear to want freedom (or perhaps she even lacks the cognitive ability to conceptualize freedom) by providing this contradictory relationship, based on the idea of freedom, Henríquez calcifies the paradoxical ideas of agency in the novel.

While on the surface, it may appear to be a typical and straightforward narrative of the struggle of immigration  \textbf{\textbf{The Book Of Unknown Americans} }is a rich and complex text. Brimming with life, Henríquez’s novel reveals itself to be a multifaceted exploration of the contradiction inherent to the immigrant experience in the United States. Through metaphors of vehicle ownership, freudian conflict, Henríquez constructs a narrative that not only gives voice to Latinx immigrant communities, but also interrogates and destabilizes the foundational myths of American liberty and self-determination.  

Ultimately, \textbf{The Book Of Unknown Americans} functions as both narrative and critique. Henríquez’s writing invites the reader to empathize with her characters’ dreams and losses while simultaneously recognizing the systemic forces that define and limit their existence. In doing so, she destabilizes the mythos of American freedom as not only incomplete but also exclusionary, built on the backs of those denied participation in its promises. The novel challenges what it means to be free and who is truly allowed to be an agent in the American story.

\pagebreak

\printbibliography

\end{mla}

\end{document}