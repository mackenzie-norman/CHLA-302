\documentclass{article}
\usepackage{ifpdf} 
\usepackage[style =mla, backend=biber]{biblatex}
\addbibresource{research.bib}
\usepackage{mla}
\title{Midterm Essay}
\author{Mackenzie Norman}
\begin{document}
\begin{mla}{Mackenzie}{Norman}{Herrera}{CHLA302}{\today}{Questions of Agency in The Book of Unknown Americans}

\noindent\textit{``Alas", said the mouse, ``the whole world is growing smaller every day. At the beginning it was so big that I was afraid, I kept running and running, and I was glad when I saw walls far away to the right and left, but these long walls have narrowed so quickly that I am in the last chamber already, and there in the corner stands the trap that I am running into." \linebreak \linebreak ``You only need to change your direction," said the cat, and ate it up.''} (\cite{eine-fable})

\section*{Introduction}
Freedom is a central tenet to the so-called American Dream. However, agency - or the ability for your choices to change the world around you - is what I like to think of as the 'why' behind this freedom. If your choices mean nothing then are you free, and if you are not free then are you even american? These difficult questions are what Christina Henríquez asks in The Book of Unknown Americans. Situated squarely in the genre defined as ``Testimonio''(\cite{testimonio}), The Book of Unknown Americans closely follows two immigrant families -- the Rivieras and the Toros -- living in an Apartment complex in Maryland. Narrated primarily by Alma Riveria, the matriarch of the Riviera family and Mayor Toro, the adolescent son. The story begins with the arrival of the Rivieras in an unknown apartment complex, where it is quickly exposed that they speak no English between the three of them. The reader is then introduced to the Toros first through Mayor and then Rafael briefly. Quickly the Toros and the Riverias become friends, with Mayor developing a crush on the daughter Maribel, this is not without reservation, since there is ``something wrong with her'' . As the novel progresses, Mayor and Maribel romance blossoms, with him even defending her from the looming villain Garret. However, things quickly deteriorate, as america plunges into recession and both patriarchs lose their jobs, which causes the non-citizen Riviera's to lose their work visa, rendering them illegal immigrants. During this time Mayor is also grounded for lying about his role on the soccer team and more importantly for Quisqueya (the perennially nosey neighbor) telling his mother that she saw him with semen on his pants leaving the car with Maribel. The novel then rachets up its pace considerably, Mayor steals his parents new car to take Maribel to see the coast while its snowing. This causes alarm with both the Riveria's and the Toros. In this panic, Alma confesses to Arturo that she saw Garret attempting to assault Maribel. Arturo, going alone, rides the bus over to search for Maribel where he is senselessly murdered by Garret's father. The story focuses thematically on guilt, the american dream and love. However Henríquez also spends considerable time questioning ideas of agency. This essay will elucidate how Henríquez expresses that inherent to the american immigrant experience are idiosyncratic and often contradictory feelings of agency.

\section*{Vehicles in The Book of Unknown Americans}
The car ``became equated early in the American cultural imagination with ... personal reinvention and self-determination'' (\cite{Uhlman2015-qx}) A symbol of freedom ``Those who could control their own movement were deemed self-sufficient, independent agents'' (\cite{Uhlman2015-qx}) Much of the initial conflict in the story is based on the dependence of public transit - Maribel's initial assault at the hands of Garret, is because A. Alma gets on the wrong bus and gets lost trying to get home in time to pick Maribel up from the bus, and B. because of the unyielding timetable of the school bus that drops her off. So it represents a change when the (citizen) Toros, after receiving money from an aunts alimony, purchase a Volkswagen. After purchasing it we immediately see the first instance of the double bind. Despite the new symbolic freedom the Toro's appear have acquired, Rafael drives far slower than the other cars going ``twenty-five'' (\cite{Henriquez2014-sh}, 164) in a ``fifty'' (\cite{Henriquez2014-sh}, 164), symbolizing that despite their new vehicle; they are still stuck halfway. Continuing this Rafael explains
\paragraph{}
\noindent ``You don't understand,'' my dad said. ``They stop you''
\noindent\linebreak
``Who? What are you talking about?'' my mom asked.
\noindent\linebreak
``Thats why I was being cautious.''
\noindent\linebreak
``Who stops you?''
\noindent\linebreak
``The police. If you're white, or maybe Oriental, they let you drive however you want. But if you're not, they stop you.''
\paragraph{}
Here, through Rafael's plaintive speech Henríquez lays bare the inherent contradiction in car ownership, that is so symbolic of the double-bind of agency in america. Even though he is promised ``The unrestrained capacity to move'' (\cite{Uhlman2015-qx}, 1) he must do so at half the pace of his white counterparts, and as such his capacity to move is restrained and so Henríquez raises a question of if he is an ``independent agent''(\cite{Uhlman2015-qx}, 1).

Henríquez furthers this question with Mayors `stealing' his fathers car to take Maribel on a date to see the Ocean. We know from the beginning Mayor is a character acutely aware of the ``double bind''.  So when he takes his fathers car, the reader is acutely aware of the apparent impending unavoidable danger. However in Mayors mind, he can only think of one thing highlighted here.

``Maribel and I deserved to be together and she deserved to see the snow if she wanted to and nobody was going to hold us back. I was her one chance. I wanted to give her the thing that it seemed like everyone else wanted to keep from her: freedom '' (\cite{Henriquez2014-sh}, )

Stealing the car sets into motion the events that lead to the hate crime causing Arturo's death, described by Mayor as either ``completely random, just something that happened'' (\cite{Henriquez2014-sh}, 261), or pre-determined ``on a path'' (\cite{Henriquez2014-sh}, 261) I do not see this as shirking responsibility, but rather the realization that, even in his pursuit (and sucsess) in acheiving freedom, there still lies a contradiction. In america, you can never have the freedom to change things unless ``you're white, or maybe Oriental''. The only agent is the one with the power to kill, the white man. `` The Rule of Law is the Rule of Force.''(\cite{loves-bdy}, 18)

\section*{Mayors conflict with his Father}

The primary contradiction of agency in the novel is Mayors conflict with his father. Mayor says he ``wasn’t allowed to claim the thing I felt and I didn’t feel the thing I was supposed to claim''(\cite{Henriquez2014-sh}, 109). This is one of the first acknowledgements of freudian psychology in the text, 
Mayor , ``crucified by the contradictory commands issuing from the Freudian super-ego, which says both "thou shalt be like the father," and "thou shalt not be like the father" (\cite{loves-bdy}, 6 ). The car - a symbol of manhood; ``Do you have anything Italian?''(\cite{Henriquez2014-sh}, 161) asks Rafael. Affirming the ``equation of [foreign luxury vehicles] and true manhood''(\cite{Uhlman2015-qx}, 19). After Mayor brings Maribel into the car, he quickly adopts it as his own manhood, likening her actions ``she grazed her fingertips across the center console, her nails scraping the hard plastic, I ... felt the blood pounding'' (\cite{Henriquez2014-sh}, 173) to her grazing his own `manhood'. ``and then my pants were damp and warm'' (\cite{Henriquez2014-sh}, 175). This equivalency starts Mayor and his father on a collision course, ``Sonship and brotherhood are espoused against fatherhood'' (\cite{loves-bdy}, 6). This conflict escalates as Rafael symbolically castrates Mayor as punishment for using his manhood, ``You're not going to see her again''. Creating  ``simultaneously a social reality and a legal impossibility – a subject barred from citizenship and without rights'' (\cite{Lutes_Travis_2021}). This castration is symbolic, but also furthers the idea of being free, yet unable to enact change - a castrated man cannot reproduce and as such cannot become a patriarch - ``a look of envy. Dreams of possession'' (\cite{wretched}, 4). Mayor yearns, not to be free, but to whole, wholly able to affect his world - ``give her the thing that it seemed like everyone else wanted to keep from her: freedom'' (\cite{Henriquez2014-sh}, )

\section*{Maribel’s role as a non agent}


``Growing up consists in finding new toys,new symbolic equivalents''(\cite{loves-bdy}, 37)


\section*{Guilt as a manifestation of agency in The Book of Unknown Americans}
Much of the conflict in The Book of Unknown Americans is about guilt and fault. (Who is at fault for maribel accident, who is at fault for Arturos death). Drilling down past guilt, a larger question is whether or not the characters are agent enough to be at fault. 
\section*{Jeffersonian ideals and Masculine and Feminine agency}
Much of the american masculine ethos is derived from Jeffersonian ideals - which encompasses much of the individual agency we see in The Book of Unknown Americans (espcially on the masculine side).  Continuing to contrast how the men and women view individualism and their agency.

\section*{Henríquez's insitince on cartesian dualism.}
Henríquez loves to use body symbolism which become more and more apparent after several readings. While Mayor reflects on his role in Arturos death he thinks: ``All these different veins, but who knew which one led to the heart?''. Alma describes the doctor coming from ``The bowels of the hospital'' to tell them their daughter is brain damaged. The commonality between these events is they are uncontrollable tragedies. 

\section*{Adolescence as a proving grounds for agency}
``The fraternity, or club, or secret society strives to put asunder what is joined in the family—male and female, parent and child. In primitive secret societies, in puberty rites, in \textit{Altersklassen und Mannerbunde}, the persistent tendency is to separate the sexes and the generations; to form homosexual and coeval groupings. Besides the natural union of the sexes in the family of which Aristotle speaks, there is also unconscious hostility between the sexes; ``an archaic reaction of enmity''; taboos which prescribe sexual separation, mutual avoidance; the castration complex."

Focusing on Mayor and how he (as any bildungsroman hero does) finds his own agency, and the responsibility that comes with being able to control your life (guilt)






\printbibliography

\end{mla}

\end{document}