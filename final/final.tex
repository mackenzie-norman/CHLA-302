\documentclass{article}
\usepackage{ifpdf} 
\usepackage[backend=biber]{biblatex}
\addbibresource{research.bib}
\usepackage{mla}
\title{Midterm Essay}
\author{Mackenzie Norman}
\begin{document}
\begin{mla}{Mackenzie}{Norman}{Herrera}{CHLA302}{\today}{Questions of Agency in The Book of Unknown Americans}

\noindent\textit{`` ``Alas", said the mouse, ``the whole world is growing smaller every day. At the beginning it was so big that I was afraid, I kept running and running, and I was glad when I saw walls far away to the right and left, but these long walls have narrowed so quickly that I am in the last chamber already, and there in the corner stands the trap that I am running into." \linebreak \linebreak ``You only need to change your direction," said the cat, and ate it up.''}

\section*{Introduction}
Freedom is a central tenet to the so-called American Dream. However, agency - or the ability for your choices to change the world around you - is what I like to think of as the 'why' behind this freedom. If your choices mean nothing then are you free, and if you are not free then are you even american? These difficult questions are what Christina Henríquez asks in The Book of Unknown Americans. Situated squarely in the genre defined by Cantu as ``Testimonio''\cite{testimonio}, The Book of Unknown Americans closely follows two immigrant families -- the Rivieras and the Toros -- living in an Apartment complex in Maryland. Narrated primarily by Alma Riveria, the matriarch of the Riviera family and Mayor Toro, the adolescent son. The story focuses thematically on guilt, the american dream and love. However Henríquez also spends considerable time questioning ideas of agency. This essay will elucidate how Henríquez expresses that inherent to the american immigrant experience are idiosyncratic and often contradictory feelings of agency.

\section*{Vehicles in The Book of Unknown Americans}
The car is inherent to america 

\section*{Guilt as a manifestation of agency in The Book of Unknown Americans}
Much of the conflict in The Book of Unknown Americans is about guilt and fault. (Who is at fault for maribel accident, who is at fault for Arturos death). Drilling down past guilt, a larger question is whether or not the characters are agent enough to be at fault. 
\section*{Jeffersonian ideals and Masculine and Feminine agency}
Much of the american masculine ethos is derived from Jeffersonian ideals - which encompasses much of the individual agency we see in The Book of Unknown Americans (espcially on the masculine side).  Continuing to contrast how the men and women view individualism and their agency.
\section*{Mayors Oedipal conflict with his Father}

\section*{Henríquez's insitince on cartesian dualism.}
Henríquez loves to use body symbolism which become more and more apparent after several readings. While Mayor reflects on his role in Arturos death he thinks: ``All these different veins, but who knew which one led to the heart?''. Alma describes the doctor coming from ``The bowels of the hospital'' to tell them their daughter is brain damaged. The commonality between these events is they are uncontrollable tragedies. 

\section*{Adolescence as a proving grounds for agency}
``The fraternity, or club, or secret society strives to put asunder what is joined in the family—male and female, parent and child. In primitive secret societies, in puberty rites, in \textit{Altersklassen und Mannerbunde}, the persistent tendency is to separate the sexes and the generations; to form homosexual and coeval groupings. Besides the natural union of the sexes in the family of which Aristotle speaks, there is also unconscious hostility between the sexes; ``an archaic reaction of enmity''; taboos which prescribe sexual separation, mutual avoidance; the castration complex."

Focusing on Mayor and how he (as any bildungsroman hero does) finds his own agency, and the responsibility that comes with being able to control your life (guilt)


\section*{Maribel’s role as a non agent}


\end{mla}
\end{document}